\documentclass{article}
\usepackage[utf8]{inputenc}
\usepackage{amsmath}
\usepackage{amssymb}
\usepackage{amsthm}
\usepackage{bm}

\title{Computational Physics (PHYS514) Final Project}
\author{Mehmet Eren Erken \\
Koç University}
\date{}

\begin{document}

\maketitle

\section*{Newton}

This part gives calculations of the structures of various types of stars in Newtonian gravity, general relativity (GR), and alternative theories of gravity which try to surpass GR.

\section{From the stellar structure equations to the Lane–Emden equation}

\subsection{Stellar structure in Newtonian gravity}

We start with the standard Newtonian equations for hydrostatic equilibrium in a spherically symmetric star:

1. **Mass continuity:**
   \[
   \frac{dm}{dr} = 4\pi r^2 \rho(r),
   \]
2. **Hydrostatic equilibrium:**
   \[
   \frac{dp}{dr} = -\frac{G m(r) \rho(r)}{r^2}.
   \]

Here:
- \(m(r)\) is the mass enclosed within radius \(r\),
- \(\rho(r)\) is the mass density,
- \(p(r)\) is the pressure,
- \(G\) is the gravitational constant.

\subsection{Polytropic equation of state}

We then close the system using a polytropic equation of state:

\[
p = K \rho^\gamma = K \rho^{1 + \tfrac{1}{n}},
\]

where
- \(K\) is a constant (related to the microphysics of the stellar material),
- \(n\) is called the polytropic index,
- \(\gamma = 1 + \frac{1}{n}\).

\subsection{Dimensionless variables}

To transform these ODEs into the Lane–Emden equation, one introduces dimensionless variables that factor out the central values. Let
\[
\rho(r) = \rho_c \theta^n(\xi),
\]
where \(\rho_c = \rho(0)\) is the central density, and define
\[
r = a \xi,
\]
for some length scale \(a\) that will be determined shortly. The quantity \(\theta(\xi)\) is a dimensionless function satisfying \(\theta(0) = 1\).  

From the polytropic EOS, the central pressure is
\[
p_c = K \rho_c^\gamma.
\]
A convenient choice for \(a\) is
\[
a^2 = \frac{(n+1) K \rho_c^{\tfrac{1}{n}}}{4\pi G}.
\]

Substituting \(\rho(r) = \rho_c \theta^n(\xi)\) into the equations and using the chosen \(a\), one arrives at a single, dimensionless ODE for \(\theta(\xi)\).

\subsection{The Lane–Emden equation}

The final result of this procedure is the Lane–Emden equation of index \(n\):
\[
\boxed{
\frac{1}{\xi^2} \frac{d}{d\xi} \left( \xi^2 \frac{d\theta}{d\xi} \right) + \theta^n = 0.
}
\]

The corresponding boundary conditions at the center (\(\xi=0\)) are:
1. \(\theta(0) = 1\), since \(\rho(0) = \rho_c\),
2. \(\theta'(0) = 0\), for regularity at the origin.

Hence, we have everything that defines the Lane–Emden problem:
\[
\begin{cases}
\frac{1}{\xi^2} \frac{d}{d\xi} \left( \xi^2 \frac{d\theta}{d\xi} \right) + \theta^n = 0, \\[6pt]
\theta(0) = 1, \quad \theta'(0) = 0.
\end{cases}
\]



\end{document}
