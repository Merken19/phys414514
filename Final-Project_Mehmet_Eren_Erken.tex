\documentclass[12pt]{article}

\usepackage{amsthm}
\usepackage{hyperref}
\usepackage{bm}
\usepackage[utf8]{inputenc}
\usepackage{amsmath, amssymb}
\usepackage{titlesec}

% Change section numbering to letters
\renewcommand{\thesection}{\Alph{section}}
\renewcommand{\thesubsection}{\thesection.\arabic{subsection}}
\renewcommand{\thesubsubsection}{\thesubsection.\arabic{subsubsection}}

% Styling sections and subsections
\titleformat{\section}{\Large\bfseries}{\thesection.}{1em}{}
\titleformat{\subsection}{\large\bfseries}{\thesubsection.}{1em}{}
\titleformat{\subsubsection}{\normalsize\bfseries}{\thesubsubsection.}{1em}{}

% Adjust spacing
\usepackage{setspace}
\setstretch{1.5}



\title{Computational Physics (PHYS414/514) \\
Final Project}
\author{Mehmet Eren Erken \\
Koç University}
\date{}

\begin{document}

\maketitle

\clearpage % Ensures the content after this starts on a new page

\section*{Newton}

This part gives calculations of the structures of various types of stars in Newtonian gravity, general relativity (GR), and alternative theories of gravity which try to surpass GR.

\section{Lane–Emden Equation}

\subsection{Stellar Structure in Newtonian Gravity}

We start with the standard Newtonian equations for hydrostatic equilibrium in a spherically symmetric star:

1. \textbf{Mass Continuity}:
   \[
   \frac{dm}{dr} = 4\pi r^2 \rho(r),
   \]

2. \textbf{Hydrostatic Equilibrium}:
   \[
   \frac{dp}{dr} = -\frac{G m(r) \rho(r)}{r^2}.
   \]

where:
\begin{itemize}
    \item \( m(r) \): mass enclosed within radius \( r \),
    \item \( \rho(r) \): mass density,
    \item \( p(r) \): pressure,
    \item \( G \): gravitational constant.
\end{itemize}

\subsection{Polytropic Equation of State}

We then close the system using a polytropic equation of state:

\[
p = K \rho^\gamma = K \rho^{1 + \tfrac{1}{n}},
\]

where:
\begin{itemize}
    \item \( K \): constant related to the microphysics of the stellar material,
    \item \( n \): polytropic index,
    \item \(\gamma = 1 + \frac{1}{n}\): adiabatic index.
\end{itemize}

\subsection{Dimensionless Variables}

To simplify the equations, we introduce dimensionless variables (e.g., Lane–Emden variables):

\[
\theta = \left(\frac{\rho}{\rho_c}\right)^{1/n}, \quad \xi = \frac{r}{r_0},
\]

where:
\begin{itemize}
    \item \(\rho_c\): central density,
    \item \(r_0 = \sqrt{\frac{(n+1)K\rho_c^{1/n - 1}}{4\pi G}}\): scaling factor for radius.
\end{itemize}

Substituting \(\rho(r) = \rho_c \theta^n(\xi)\) into the equations and using \(r_0\), we transform the ODEs into the Lane–Emden equation.

\subsection{The Lane–Emden Equation}

The final result of this procedure is the Lane–Emden equation of index \(n\):
\[
\boxed{
\frac{1}{\xi^2} \frac{d}{d\xi} \left( \xi^2 \frac{d\theta}{d\xi} \right) + \theta^n = 0.
}
\]

The corresponding boundary conditions at the center (\(\xi=0\)) are:
1. \(\theta(0) = 1\), since \(\rho(0) = \rho_c\),
2. \(\theta'(0) = 0\), for regularity at the origin.

Thus, the Lane–Emden problem is defined by:
\[
\begin{cases}
\frac{1}{\xi^2} \frac{d}{d\xi} \left( \xi^2 \frac{d\theta}{d\xi} \right) + \theta^n = 0, \\[6pt]
\theta(0) = 1, \quad \theta'(0) = 0.
\end{cases}
\]

\subsection{Series Expansion at the Center}

We verify the regularity condition near \(\xi=0\) by performing a power-series expansion. Assume:
\[
\theta(\xi) 
= 1 + a_2 \xi^2 + a_4 \xi^4 + \dots
\]

Plugging this into the Lane–Emden equation yields the coefficients \(a_2, a_4, \dots\). A calculation from \href{Newton.ipynb}{Newton.ipynb part A.1} gives:
\[
\theta(\xi)
= 1 - \frac{1}{6}\xi^2 + \frac{n}{120}\xi^4 - \cdots,
\]

confirming \(\theta'(0) = 0\).

\subsubsection{Solving the Lane–Emden equation for \(n=1\)}

When \(n = 1\), the Lane–Emden equation becomes
\[
\frac{1}{\xi^2}
\frac{d}{d\xi}
\left(
\xi^2 \frac{d\theta}{d\xi}
\right)
+ \theta = 0,
\]
with \(\theta(0)=1\) and \(\theta'(0)=0\).

This ODE can be solved analytically. For \(n=1\), the solution is:
\[
\theta(\xi) = \frac{\sin \xi}{\xi}.
\]

Indeed, one checks that
\[
\theta(0) = \lim_{\xi \to 0} \frac{\sin \xi}{\xi} = 1,
\quad
\theta'(0) = 0,
\]
and it satisfies the differential equation upon direct substitution. The solution can also be solved using sympy, which is demonstrated in \href{Newton.ipynb}{Newton.ipynb part A.2}.

\subsection{Defining the stellar surface and total mass}

\subsubsection{Surface of the polytrope}

Because \(\rho(r) \propto \theta^n(\xi)\), the surface of the star is (by definition) at the first positive \(\xi = \xi_n\) such that 
\[
\theta(\xi_n) = 0.
\]
Then the physical radius of the star is 
\[
R = a \xi_n.
\]

For \(n=1\), from \(\sin(\xi_n)/\xi_n = 0\), the first positive root is \(\xi_n = \pi\). For other integer \(n\), one must solve numerically or use known special-function expansions.

\subsubsection{Total Mass of the Star}

To find the total mass of the star, we start with the dimensionless form of the mass-continuity equation:
\[
\frac{dm}{d\xi} = \xi^2 \theta^n.
\]

The total mass \(M\) of the star is the mass enclosed at the surface, which corresponds to \(\xi = \xi_n\), where \(\xi_n\) is the dimensionless radius at which \(\theta(\xi_n) = 0\). Integrating from the center (\(\xi = 0\)) to the surface (\(\xi = \xi_n\)):
\[
M = 4\pi \rho_c a^3 \int_0^{\xi_n} \xi^2 \theta^n \, d\xi.
\]

Using integration by parts to simplify the integral, we start by rewriting the integrand:
\[
\int_0^{\xi_n} \xi^2 \theta^n \, d\xi = \left[ -\xi^2 \theta' \right]_0^{\xi_n} + \int_0^{\xi_n} 2\xi \theta' \, d\xi.
\]

From the Lane–Emden equation:
\[
\frac{1}{\xi^2} \frac{d}{d\xi} \left( \xi^2 \frac{d\theta}{d\xi} \right) + \theta^n = 0,
\]
we know:
\[
\xi^2 \theta^n = -\frac{d}{d\xi} \left( \xi^2 \frac{d\theta}{d\xi} \right).
\]

This implies that the integral of \(\xi^2 \theta^n\) simplifies directly using the surface boundary conditions:
\[
\int_0^{\xi_n} \xi^2 \theta^n \, d\xi = -\left[\xi^2 \frac{d\theta}{d\xi}\right]_0^{\xi_n}.
\]

At the center, \(\xi = 0\), the regularity condition ensures that \(\xi^2 \theta' \to 0\). Therefore:
\[
M = 4\pi \rho_c a^3 \left[ -\xi_n^2 \theta'(\xi_n) \right].
\]

Rewriting \(a = R / \xi_n\), where \(R\) is the physical radius of the star, we obtain:
\[
M = 4\pi \rho_c R^3 \left[ -\frac{\theta'(\xi_n)}{\xi_n} \right].
\]

\subsection{Mass–radius relation for polytropes}

Finally, if one fixes the same polytropic index \(n\) (i.e., all stars in the family have the same value of \(n\)) but allows different central densities \(\rho_c\), then the scaling of \(a\) with \(\rho_c\) implies a specific power-law relation between \(M\) and \(R\).

\subsubsection{How \(a\) depends on \(\rho_c\)}

Recall
\[
a^2 
= \frac{(n+1)K\rho_c^{1/n}}{4\pi G}
\quad\Longrightarrow\quad
a \propto \rho_c^{1/(2n)}.
\]
Hence
\[
R = a \xi_n \propto \rho_c^{1/(2n)}.
\]

\subsubsection{How \(M\) Depends on \(R\)}

Substituting this into the expression for \(M\):
\[
M \propto \rho_c \cdot R^3 \propto R^{-2n} \cdot R^3.
\]

Simplifying gives:
\[
M \propto R^{3 - 2n}.
\]

Using the polytropic equation of state \(p \propto \rho^{1 + 1/n}\) leads to:
\[
M \propto R^{\frac{3 - n}{1 - n}}.
\]


\subsubsection{Finding the Constant of Proportionality}

To determine the constant of proportionality, we start by expressing the central density \(\rho_c\) in terms of other variables. Recall the expression for the Lane–Emden scaling factor \(a\):
\[
a^2 = \frac{(n+1)K\rho_c^{1/n}}{4\pi G}.
\]

Solving for \(\rho_c\):
\[
\rho_c = \left(\frac{4\pi G}{(n+1)K}\right)^n a^{-2n}.
\]

Substituting \(a = R / \xi_n\), we express \(\rho_c\) as:
\[
\rho_c = \left(\frac{4\pi G}{(n+1)K}\right)^n \left(\frac{R}{\xi_n}\right)^{-2n}.
\]

Now, substitute this expression for \(\rho_c\) into the total mass formula:
\[
M = 4\pi \rho_c R^3 \left(-\frac{\theta'(\xi_n)}{\xi_n}\right).
\]

Expanding \(\rho_c\) explicitly:
\[
M = 4\pi \left(\frac{4\pi G}{(n+1)K}\right)^n \left(\frac{R}{\xi_n}\right)^{-2n} R^3 \left(-\frac{\theta'(\xi_n)}{\xi_n}\right).
\]

Simplify the powers of \(R\) to consolidate the mass-radius relation:
\[
M = \left(-4\pi \right) \left(\frac{4\pi G}{(n+1)K}\right)^n \xi_n^{1-n} \left(-\theta'(\xi_n)\right) R^{3-n}.
\]

Factor out the terms that depend only on \(n\), \(K\), and \(G\):
\[
M = C(n) K^{n/(n-1)} G^{-n/(n-1)} R^{\frac{3-n}{1-n}},
\]

where the dimensionless constant \(C(n)\) is:
\[
C(n) = 4\pi \left(4\pi\right)^n \frac{\xi_n^{1-n} \left(-\theta'(\xi_n)\right)}{(n+1)^n}.
\]

\end{document}
