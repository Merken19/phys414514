\documentclass{article}
\usepackage[utf8]{inputenc}
\usepackage{amsmath}
\usepackage{amssymb}
\usepackage{amsthm}
\usepackage{hyperref}
\usepackage{bm}

\title{Computational Physics (PHYS514) Final Project}
\author{Mehmet Eren Erken \\
Koç University}
\date{}

\begin{document}

\maketitle

\section*{Newton}

This part gives calculations of the structures of various types of stars in Newtonian gravity, general relativity (GR), and alternative theories of gravity which try to surpass GR.

\section{From the Stellar Structure Equations to the Lane–Emden Equation}

\subsection{Stellar Structure in Newtonian Gravity}

We start with the standard Newtonian equations for hydrostatic equilibrium in a spherically symmetric star:

1. \textbf{Mass Continuity}:
   \[
   \frac{dm}{dr} = 4\pi r^2 \rho(r),
   \]

2. \textbf{Hydrostatic Equilibrium}:
   \[
   \frac{dp}{dr} = -\frac{G m(r) \rho(r)}{r^2}.
   \]

where:
\begin{itemize}
    \item \( m(r) \): mass enclosed within radius \( r \),
    \item \( \rho(r) \): mass density,
    \item \( p(r) \): pressure,
    \item \( G \): gravitational constant.
\end{itemize}

\subsection{Polytropic Equation of State}

We then close the system using a polytropic equation of state:

\[
p = K \rho^\gamma = K \rho^{1 + \tfrac{1}{n}},
\]

where:
\begin{itemize}
    \item \( K \): constant related to the microphysics of the stellar material,
    \item \( n \): polytropic index,
    \item \(\gamma = 1 + \frac{1}{n}\): adiabatic index.
\end{itemize}

\subsection{Dimensionless Variables}

To simplify the equations, we introduce dimensionless variables (e.g., Lane–Emden variables):

\[
\theta = \left(\frac{\rho}{\rho_c}\right)^{1/n}, \quad \xi = \frac{r}{r_0},
\]

where:
\begin{itemize}
    \item \(\rho_c\): central density,
    \item \(r_0 = \sqrt{\frac{(n+1)K\rho_c^{1/n - 1}}{4\pi G}}\): scaling factor for radius.
\end{itemize}

Substituting \(\rho(r) = \rho_c \theta^n(\xi)\) into the equations and using \(r_0\), we transform the ODEs into the Lane–Emden equation.

\subsection{The Lane–Emden Equation}

The final result of this procedure is the Lane–Emden equation of index \(n\):
\[
\boxed{
\frac{1}{\xi^2} \frac{d}{d\xi} \left( \xi^2 \frac{d\theta}{d\xi} \right) + \theta^n = 0.
}
\]

The corresponding boundary conditions at the center (\(\xi=0\)) are:
1. \(\theta(0) = 1\), since \(\rho(0) = \rho_c\),
2. \(\theta'(0) = 0\), for regularity at the origin.

Thus, the Lane–Emden problem is defined by:
\[
\begin{cases}
\frac{1}{\xi^2} \frac{d}{d\xi} \left( \xi^2 \frac{d\theta}{d\xi} \right) + \theta^n = 0, \\[6pt]
\theta(0) = 1, \quad \theta'(0) = 0.
\end{cases}
\]

\subsection{Series Expansion at the Center}

We verify the regularity condition near \(\xi=0\) by performing a power-series expansion. Assume:
\[
\theta(\xi) 
= 1 + a_2 \xi^2 + a_4 \xi^4 + \dots
\]

Plugging this into the Lane–Emden equation yields the coefficients \(a_2, a_4, \dots\). A calculation from \href{Newton.ipynb}{Newton.ipynb part A} gives:
\[
\theta(\xi)
= 1 - \frac{1}{6}\xi^2 + \frac{n}{120}\xi^4 - \cdots,
\]

confirming \(\theta'(0) = 0\).

\end{document}
